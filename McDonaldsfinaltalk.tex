\documentclass[compress,handout,10pt]{beamer}

\newlength{\wideitemsep}
\setlength{\wideitemsep}{\itemsep}
\addtolength{\wideitemsep}{100pt}
\let\olditem\item
\renewcommand{\item}{\setlength{\itemsep}{0.5\baselineskip}\olditem}

\usetheme{Singapore}
\usecolortheme{lily}
\usefonttheme[onlymath]{serif}

\usepackage{float}
\floatstyle{boxed}
\usepackage{colortbl}
\usepackage{mathpazo}
\usepackage{graphicx}
\usepackage{movie15}
\usepackage{bm}
\usepackage{verbatim}
\usepackage{comment}
\usepackage{caption}
\usepackage{subcaption}
\captionsetup[subfigure]{labelformat=empty}
\captionsetup[figure]{labelformat=empty}

\newcommand{\mygreen}{\color{green!50!black}}
\newcommand{\myblue}{\color{blue}}
\newcommand{\myred}{\color{red}}
\newcommand{\mycolor}{\color{red}{c}\color{blue}{o}\color{green}{l}\color{orange}{o}\color{cyan}{r}}
\newcommand{\mysize}{\scriptsize{s}\small{i}\normalsize{z}\Large{e}}
\newcommand{\myshape}{\textcircled{s}\textit{h}\texttt{a}\textsf{p}\textsc{e}}

\xdefinecolor{titlecolor}{rgb}{.855,.647,.125}
\setbeamercolor{frametitle}{fg=titlecolor}
\setbeamerfont{frametitle}{series=\bfseries}
\setbeamercolor{normal text in math text}{parent=math text}

\setbeamertemplate{navigation symbols}{} %gets rid of navigation symbols
\setbeamertemplate{footline}[frame number]
\beamertemplateshadingbackground{blue!5}{yellow!10}

\title{{\color{blue} \LARGE How much Ice do You need?\newline} }

\subtitle{{\color{red} \large Final Presentation} }

\author{ 
%    \vspace{5pt}
    {\bf{Participant:}} \\ 
Joyce Tan \\ 
Yen Theng Tan \\
    \vspace{5pt}
} 
\institute{JHU AMS 2012 FALL}

\date{\mygreen Last Complied on \today} 

\begin{document}

\begin{frame}[plain]
    \titlepage
\end{frame}


\begin{frame}
    \frametitle{Outline}
      \tableofcontents
\end{frame}

\section{Introduction}
\subsection{Sponsor}
\begin{frame}
    \frametitle{Sponsor: McDonald's Corporation}
    \begin{itemize}
     \item McDonald's Corporation is the world's largest chain of hamburger fastfood restaurants, serving around 68 million customers daily in 119 countries.
\item Mcdonald's primarily sells hamburgers, cheeseburgers, chicken, French fries, breakfast items, soft drinks, milkshakes and desserts. 
    \end{itemize}
\end{frame}

\begin{frame}
    \frametitle{Sponsor: McDonald's Corporation}
    \begin{itemize}
         	\item In response to healthier consumer taste, the company has expanded its menu to include salads, wraps, smoothies and fruits.
	 \item No meal is complete without a drink; and from Diet Coke to low-fat milk to fresh-brewed, hot coffee, McDonald's serves many different varieties of beverages
    \end{itemize}
\end{frame}

\subsection{Problem Statement}
\begin{frame}
    \frametitle{Problem Statement}
     \begin{itemize}
         \item Selling soft drinks is a complement to any meal that a customer purchases at McDonald's.
\item However, the server is not accustomed to putting much thought in measuring the amount of ice put in the cup.
\item This often results in a overly diluted, or overly cold drink for the customer. This is likely to lower overall customer satisfaction, since a drink is a significant complement to a meal. 
\item Thus, customers are likely to appreciate if the right amount of ice was added for optimal satisfaction.
     \end{itemize}
\end{frame}

\begin{frame}
    \frametitle{Problem Statement}
     \begin{itemize}
      \item To further define this problem, the exogenous variables are the proportion of ice to put in a drink. 
\item The endogenous variable would be the resulting temperature and concentration of the drink, as we are assuming that a customer's satisfaction is affected only by the temperature and concentration of the drink.
     \end{itemize}
\end{frame}

\subsection{Deliverables}
\begin{frame}
    \frametitle{Deliverables - From Team to Sponsor}


\begin{itemize}
    \item A table of optimal ice proportions/ratios for each different type of soda (namely Coca Cola, Sprite, Fanta Orange, Diet Coke),
    \item Matlab code with complete set of documentations that resulting temperature and dilution based on specific heat capacities and ice proportions,
    \item Numerical experiment results reporting success rate of different ice proportions,
    \item Technical report and presentations summarizing the work. 
\end{itemize}


\end{frame}

\begin{frame}
    \frametitle{Deliverables - From Sponsor to Team}

\begin{itemize}
    \item Sufficient supply of the 4 different sodas we are concentrating on,
    \item Computing resources,
    \item Timely responses to inquiries.
\end{itemize}
\end{frame}

\subsection{Timeline}
\begin{frame}
    \frametitle{Timeline}
\begin{itemize}
    \item Work Statement due date, Sep 28, 2012,
    \item Midterm Presentation due date, Oct 17, 2012,
    \item Progress Report due date, Oct 26, 2012,
    \item Final Presentation due date, Nov 6, 2012,
    \item Final Report due date, Nov 30, 2012.
\end{itemize}
\vspace{6pt} Most of the experiments and coding will be done from mid-October to mid-November.
\end{frame}

\section{Content}
\subsection{Approach Assumptions}
\begin{frame}
    \frametitle{Approach Assumptions}

\begin {itemize}
\item Consumer's taste depends entirely on the dilution and temperature factors.
\item Dilution and temperature of drink come hand-in-hand and rely entirely on the ice proportion.
\item Sample group accurately represents the population's preferred combinations of temperature and dilution.
\item Customer only consumes the drink after all the ice has melted.
\end{itemize}
\end{frame}

\subsection{Experimental Approach}
\begin{frame}
    \frametitle{Approach 1: Experimental}

\begin {itemize}
\item Experimenting with different types of soda - namely  McDonald's Coca Cola, Sprite, Fanta Orange, and Diet Coke.
\item Using different proportions of ice, we will then find the resulting temperature of the drink, as well as calculate the resulting dilution of the drink. 

\end{itemize}

\end{frame}

\begin{frame}
    \frametitle{Approach 1: Experimental}

\begin {itemize}
\item By experiment, we will test out which combination of temperature and dilution will yield the highest satisfaction from the test subjects.
\item We will provide 3 different cups of the same soda (different ice proportions) for the test subject to drink and they will indicate their preference. This will be repeated for 3 more days for the other 3 drinks.
\item The ice will be left in the drink for a time period of t (t=0.5mins, 2 mins, 5 mins, 30 mins). The different experiments for the time parameters will be spaced an hour apart.
\end{itemize}
\end{frame}

\begin{frame}
    \frametitle{Approach 1: Experimental}

\begin {itemize}

\item This will be a blind test and the subject will not know what ice proportions the cups A, B, C have.
\end{itemize}
\vspace{6pt}

\begin{table}[ h]
\centering
\begin{tabular}{ l | c|c|c }
  Ice Proportion & A  & B & C  \\
\hline  
t=0.5mins & & &\\ 
\hline  
t=2mins & & &\\ 
\hline  
t=5mins  & & &\\ 
\hline  
t=30mins & & &\\ 
\hline  
   
 \end{tabular}
\caption{Sample form each test subject will need to fill out for each drink}

\end{table}

\end{frame}

\subsection{Physics-based Approach}
\begin{frame}
    \frametitle{Approach 2: Physics-based}

\begin {itemize}
\item Utilizing the specific heat capacities of soda and ice (already found as specific values), we can calculate the different temperatures and dilution that the resulting drink will be.
\item Using data from the first approach, we can see how the theoretical combinations of dilution and temperatures compare with the ones in practice.
\item This will be used mainly as a support tool since it's just mathematical calculation.
\end{itemize}

\end{frame}

\subsection{Results}
\begin{frame}
    \frametitle{Results - Experimental approach}

\begin{table}[ h]
\centering
\begin{tabular}{ l || c|c|c }
  &40\% &60\% & 75\%  \\
\hline  
t=0.5 mins & 15 & 25 & 32\\ 
\hline  
t=2 mins & 14 & 24 & 34\\ 
\hline  
t=5 mins & 14 & 27 & 31\\ 
\hline  
t=30 mins & 18 & 36 & 18\\ 
\hline  
   
 \end{tabular}

\caption{Experiment results for Coke}

\end{table}
\end{frame}

\begin{frame}
    \frametitle{Results - Experimental approach}

\begin{table}[ h]
\centering
\begin{tabular}{ l || c|c|c }
  &40\% &60\% & 75\%  \\
\hline  
t=0.5 mins & 15 & 27 & 30\\ 
\hline  
t=2 mins & 20 & 19 & 33\\ 
\hline  
t=5 mins & 14 & 29 & 29\\ 
\hline  
t=30 mins & 17 & 30 & 25\\ 
\hline  
   
 \end{tabular}

\caption{Experiment results for Sprite}

\end{table}
\end{frame}

\begin{frame}
    \frametitle{Results - Experimental approach}

\begin{table}[ h]
\centering
\begin{tabular}{ l || c|c|c }
  &40\% &60\% & 75\% \\
\hline  
t=0.5 mins & 15 & 23 & 34\\ 
\hline  
t=2 mins & 19 & 23 & 30\\ 
\hline  
t=5 mins & 18 & 27 & 27\\ 
\hline  
t=30 mins & 12 & 35 & 25\\ 
\hline  
   
 \end{tabular}

\caption{Experiment results for Fanta Orange}

\end{table}
\end{frame}

\begin{frame}
    \frametitle{Results - Experimental approach}
\begin{table}[ h]
\centering
\begin{tabular}{ l || c|c|c}
  &40\% &60\% & 75\%  \\
\hline  
t=0.5 mins & 15 & 24 & 33\\ 
\hline  
t=2 mins & 21& 19 & 32\\ 
\hline  
t=5 mins & 16 & 24 & 32\\ 
\hline  
t=30 mins & 18 & 22& 32\\ 
\hline  
   
 \end{tabular}

\caption{Experiment results for Diet Coke}

\end{table}
\end{frame}

\begin{frame}
    \frametitle{Results - Physics-based approach}
\begin{table}[ h]
\centering
\begin{tabular}{ l || c|c}
 Volume of ice to volume of soda &Dilution &Temperature (Celsius) \\
\hline  
1/10 & 0.09&16.2\\ 
\hline  
1/8 & 0.11&14.3\\ 
\hline 
1/6 & 0.15&11.2\\ 
\hline 
1/5 & 0.18&8.8\\ 
\hline 
1/4 & 0.23&5.5\\ 
\hline 
   
 \end{tabular}

\caption{Calculated dilution and temperature for difference ice volumes}

\end{table}
\end{frame}
 
\subsection{Analysis}
\begin{frame}
    \frametitle{Results - Experimental approach}

\begin{table}[ h]
\centering
\begin{tabular}{ l || c|c|c||c|c }
  &40\% &60\% & 75\% &p-value &significance? \\
\hline  
t=0.5 mins & 15 & 25 & 32 & 0.047&significant\\ 
\hline  
t=2 mins & 14 & 24 & 34&0.016&significant\\ 
\hline  
t=5 mins & 14 & 27 & 31&0.037&significant\\ 
\hline  
t=30 mins & 18 & 36 & 18&0.011&significant\\ 
\hline  
   
 \end{tabular}

\caption{Experiment results for Coke}

\end{table}

\begin{itemize}
\item blah blah
\end{itemize}
\end{frame}

\begin{frame}
    \frametitle{Results - Experimental approach}

\begin{table}[ h]
\centering
\begin{tabular}{ l || c|c|c||c|c }
  &40\% &60\% & 75\% &p-value &significance? \\
\hline  
t=0.5 mins & 15 & 27 & 30&0.072&not significant\\ 
\hline  
t=2 mins & 20 & 19 & 33&0.079&not significant\\ 
\hline  
t=5 mins & 14 & 29 & 29&0.044&significant\\ 
\hline  
t=30 mins & 17 & 30 & 25&0.011&significant\\ 
\hline  
   
 \end{tabular}

\caption{Experiment results for Sprite}

\end{table}

\begin{itemize}
\item blah blah
\end{itemize}
\end{frame}

%% FANTA ORANGE
\begin{frame}
    \frametitle{Results - Experimental approach}
\begin{table}[ h]
\centering
\begin{tabular}{ l || c|c|c||c|c }
  &40\% &60\% & 75\% &p-value &significance? \\
\hline  
t=0.5 mins & 15 & 23 & 34&0.022&significant\\ 
\hline  
t=2 mins & 19 & 23 & 30&0.275&not significant\\ 
\hline  
t=5 mins & 18 & 27 & 27&0.325&not significant\\ 
\hline  
t=30 mins & 12 & 35 & 25&0.004&significant\\ 
\hline     
 \end{tabular}
\caption{Experiment results for Fanta Orange}

\end{table}

\begin{itemize}
\item blah blah
\end{itemize}
\end{frame}

%% DIET COKE
\begin{frame}
    \frametitle{Results - Experimental approach}
\begin{table}[ h]
\centering
\begin{tabular}{ l || c|c|c||c|c }
  &40\% &60\% & 75\% &p-value &significance? \\
\hline  
t=0.5 mins & 15 & 24 & 33&0.034&significant\\ 
\hline  
t=2 mins & 21& 19 & 32&0.130 &not significant\\ 
\hline  
t=5 mins & 16 & 24 & 32&0.069&not significant\\ 
\hline  
t=30 mins & 18 & 22& 32&0.115&not significant\\ 
\hline  
 \end{tabular}
\caption{Experiment results for Diet Coke}
\end{table}

\begin{itemize}
\item blah blah
\end{itemize}
\end{frame}

%% ANALYSIS 
\begin{frame}
    \frametitle{Analysis - Physics-based approach}
\begin{table}[ h]
\centering
\begin{tabular}{ l || c|c}
 Volume of ice to volume of soda &Dilution &Temperature (Celsius) \\
\hline  
1/10 & 0.09&16.2\\ 
\hline  
1/8 & 0.11&14.3\\ 
\hline 
1/6 & 0.15&11.2\\ 
\hline 
1/5 & 0.18&8.8\\ 
\hline 
1/4 & 0.23&5.5\\ 
\hline    
\end{tabular}
\caption{Calculated dilution and temperature for difference ice volumes}
\end{table}

\begin{itemize}
\item blah blah
\end{itemize}

\end{frame}

%%%%%%% CONCLUSION %%%%%%%%

% From Team to Sponsor
\section{Conclusion}
\subsection{Deliverables}
\begin{frame}
    \frametitle{Deliverables - From Team to Sponsor}
\begin{itemize}
    \item A table of optimal ice proportions/ratios for each different type of soda (namely Coca Cola, Sprite, Fanta Orange, Diet Coke),
    \item Matlab code with complete set of documentations that resulting temperature and dilution based on specific heat capacities and ice proportions,
    \item Numerical experiment results reporting success rate of different ice proportions,
    \item Technical report and presentations summarizing the work. 
\end{itemize}

\end{frame}

%% From Sponsor to Team
\begin{frame}
    \frametitle{Deliverables - From Sponsor to Team}

\begin{itemize}
    \item Sufficient supply of the 4 different sodas we are concentrating on,
    \item Computing resources,
    \item Timely responses to inquiries.
\end{itemize}
\end{frame}

\subsection{Advantages and Disadvantages}
\begin{frame}
    \frametitle{Advantages}

\begin {itemize}
\item Utilizing the specific heat capacities of soda and ice, we can calculate the different combinations of temperatures and dilution of the drink.
\item By surveying our sample group (which should be a accurate presentation of the population), we can determine which is the most popular combination of temperature and dilution and thus the optimal combination of ice proportion.
\item We are able to use physics calculations to compare the accuracy of the experiments.
\end{itemize}

\end{frame}

\begin{frame}
    \frametitle{Disadvantages}

\begin{itemize}
\item Assumption that all customers have the same taste regarding temperature and dilution is probably false, yet we only offer one optimal ice proportion for each drink.
\item Desired temperature of drink may also depend on location of branch and climate.

\end{itemize}

\end{frame}

\begin{frame}
    \frametitle{Disadvantages}

\begin{itemize}

\item Physics-based calculation might not be as accurate since it assumes that there is no inteference with the environment, which is not true in reality.
\item It is more likely that a customer starts sipping the drink once he/she gets it, rather than waiting for the ice to completely melt.
\end{itemize}

\end{frame}

\subsection{Further Recommendations}
\begin{frame}
    \frametitle{Further Recommendations}
\begin{itemize}
\item Perform experiments on different days with different climates.
\item Split sample group based on gender and age.
\item Perform experiments such that test subject starts drinking once he receives it.
\end{itemize}
\end{frame}

\end{document}
